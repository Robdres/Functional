\documentclass{beamer}
\usetheme{metropolis}
\usepackage{amsmath}
\usepackage{amssymb}
\usepackage{mathtools}
\DeclarePairedDelimiterX{\inp}[2]{\langle}{\rangle}{#1, #2}

\newcommand{\norma}[2][]{\ensuremath{\left\lVert #2 \right\rVert_{#1}}}
\title{Operador Unitario, Autoadjunto y Normal}

\newcommand{\C}{\mathbb{C}}

\date{\today}
\author{Roberto Alvarado}
\institute{Universidad San Francisco de Quito}

\begin{document}

\begin{frame}
\maketitle
\end{frame}

\begin{frame}{Autoadjunto}
  Sea un operador linear acotado $T:H\longrightarrow H$ se le
  llama autoadjunto o Hermitiano si 
  $$T=T^{*}$$
\end{frame} 

\begin{frame}{Unitario}
  Sea un operador linear acotado $T:H\longrightarrow H$ se le
  llama unitario si T es biyectivo y
  $$T^{*} = T^{-1}$$
\end{frame}
  
\begin{frame}{Normal}
  Sea un operador linear acotado $T:H\longrightarrow H$ se le
  llama normal si
  $$T^{*}T = TT^{*} $$
\end{frame}

\begin{frame}
  Para el operador adjunto
  $$\langle Tx,y\rangle = \langle x,T^{*}y\rangle$$
  Cuando se trata de un operador autoadjunto
  $$\langle Tx,y\rangle = \langle x,Ty\rangle$$
\end{frame}

\begin{frame}{Auto adjunto o Unitario implica Normal}
  Autoadjunto
  $$TT^{*}=T^{*}T$$
  $$TT=TT$$
  Unitario
  $$TT^{*}=T^{*}T$$
  $$TT^{-1}=T^{-1}T$$
  $$I=I$$
\end{frame}

\begin{frame}{Normal no implica Autoadjunto o Unitario}
  Sea $T=2iI$, el adjunto es $T^{*}=-2iT$, y el inverso es $T^{-1} =
  -\frac{1}{2}iT$ cumple que
    $$TT^{*}=T^{*}T$$
    Pero 
    $$T\neq T^{*}$$
    Y a la vez
    $$T\neq T^{-1}$$
\end{frame}
\begin{frame}{$\mathbb{C}^n$}
  Producto interior definido
  $$\inp{x}{y} = \xi_0\overline \delta_0+\dots+\xi_n\overline\delta_n$$
  en notación matricial
  $$\inp{x}{y} = x^T\overline y$$
  Sea $T: \mathbb{C}^n\longrightarrow \mathbb{C}^n$ un operador linear
  acotado. Teniendo una base para $\mathbb{C}^n$, tenemos que podemos
  representar $T$ y $T^{*}$ como matrices cuadradas de dimensión n,
  respectivamente A y B.
\end{frame}

\begin{frame}{Adjunto Hermitiano}
  $$\inp{Tx}{y} = (Ax)^T\overline y = x^TA^T\overline y$$
  $$\inp{x}{T^*y} = x^T\overline{By} $$
  $$\inp{x}{T^*y} = x^T\overline{B}\overline y$$
\end{frame}

\begin{frame}{Autoadjunto}
  $$x^TA^T\overline y= x^T\overline{B}\overline y$$
  Entonces para que sea autoadjunto
  $$B = \overline{A^{T}}$$
\end{frame}

\begin{frame}{Recuerdo de matrices}
  Hermitiana si $\overline A^T = A$\\
  Hermitiana asimétrica si $\overline A^T = A$\\
  Unitaria si $\overline A^T = A^{-1}$\\
  Normal si $A\overline A^T = \overline A^T A$\\
  \vspace{1cm}
  Simétrica si $ A^T = A$\\
  Ortogonal si $A^T = A^{-1}$

\end{frame}

\begin{frame}{Operadores y Matrices en $\C^n$}
  Matriz Hermitiana si T es auto-adjunta\\
  Matriz Unitaria si T es unitario\\
  Matriz Normal si es normal
\end{frame}

\begin{frame}{Operadores y Matrices en $\mathbb{R}^n$}
  Matriz Simétrica si T es auto adjunta\\
  Matriz Ortogonal si T es unitario\\
\end{frame}

\begin{frame}{Auto-adjunto}
  Sea $T:H\longrightarrow H$ un operador linear acotado
  \begin{itemize}
    \item Si T es autoadjunto entonces $\inp{Tx}{x}$ es real $\forall
      x \in H$
      $$\overline{\inp{Tx}{x}} = \inp{x}{Tx} = \inp{Tx}{x}$$
  \end{itemize}
\end{frame}
\begin{frame}
  \begin{itemize}
    \item Si H es complejo, y $\inp{Tx}{x}$ es real $\forall x \in H$
      entonces el operador es autoadjunto
      $$\inp{Tx}{x} = \overline{\inp{Tx}{x}} = \overline{\inp{x}{T^*x}}
      = \inp{T^*x}{x}$$
      $$T - T^* = 0$$
  \end{itemize}
\end{frame}
\begin{frame}{Autoadjunto de un producto}
  Sea S,T dos operadores lineales autoadjuntos, ST es autoadjunto si y
  solo si 
  $$ST=TS$$
\end{frame}
\begin{frame}{Demostración}
  $$ ST =(ST)^* = T^*S^* = TS$$
  $$ (ST)^* = T^*S^* = TS = ST $$
\end{frame}

\begin{frame}{Secuencia de operadores autoadjuntos}
  Sea $(T_n)$ una secuencia de de operadores lineales autoadjuntos en
  H. Supongamos que converge a T
  $$T_n\longrightarrow T~~~~\norma{T_n-T}{}\longleftarrow 0$$
  Entonces T es un operador linear acotado autoadjunto en H.
\end{frame}
\begin{frame}{Demostración}
  Tenemos que demostrar que 
  $$\norma{T-T^*}{} = 0$$
  Sabemos que
  $$\norma{T_n^*-T^*}{} = \norma{T_n-T}{}$$

\end{frame} 
\begin{frame}{Demostración}
$$\norma{T-T^*}{}\leqq \norma{T-T_n}{} +\norma{T_n-T_n^*}{}+ \norma{T_n^*-T}{}$$

$$=2\norma{T_n-T}{}\longrightarrow 0 $$
$$\norma{T-T^*}{} = 0$$
$$T=T^*$$
\end{frame}

\begin{frame}{Operador Unitario}
  Sea U,V un operador unitario sobre H.
  \begin{itemize}
    \item U es isométrico 
      $$\norma{Ux}{}^2=\inp{Ux}{Ux}=\inp{x}{U^*Ux} =
      \inp{x}{Ix}=\norma{x}{}^2$$
  \end{itemize}
\end{frame}

\begin{frame}{Operador Unitario}
  \begin{itemize}
    \item $\norma{U}{}=1$ siendo $H\neq \{0\}$
    \item $U^{-1}$ es unitario
      $$(U^{-1})^* = U^{**} = U = (U^{-1})^{-1}$$
  \end{itemize}
\end{frame}

\begin{frame}{Operador Unitario}
  \begin{itemize}
    \item UV es unitario
      UV es biyectivo
      $$(UV)^* = V^*U^* = V^{-1}U^{-1}=(UV)^{-1}$$
  \end{itemize}
\end{frame}

\begin{frame}{Operador Unitario}
  \begin{itemize}
    \item U es normal
      $$(UV)^* = V^*U^* = V^{-1}U^{-1}=(UV)^{-1}$$
  \end{itemize}
\end{frame}

\begin{frame}{Operador Unitario}
  \begin{itemize}
    \item Un operador linear acotado T en un espacio de Hilbert es
      unitario si y solo si T es isométrico y sobreyectiva
  \end{itemize}
\end{frame}
\begin{frame}{Demostración}
  $$\inp{TT^*x}{x} = \inp{Tx}{Tx} = \inp{Ix}{x} $$
  Entonces 
  $$\inp{(T^*T-I)x}{x} =0$$
  Entonces
  $$TT^* = I$$
  $\dots$
  $$T^*=T^{-1}$$
\end{frame}
\end{document}
